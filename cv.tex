% (c) 2002 Matthew Boedicker <mboedick@mboedick.org> (original author) 
% http://mboedick.org
% (c) 2003-2007 David J. Grant <davidgrant-at-gmail.com> 
% http://www.davidgrant.ca
% (c) 2008 Nathaniel Johnston <nathaniel@nathanieljohnston.com> 
% http://www.nathanieljohnston.com
%
% (c) 2012 Scott Clark <sc932@cornell.edu> cam.cornell.edu/~sc932
%
%This work is licensed under the Creative Commons 
% Attribution-Noncommercial-Share Alike 2.5 License. To view a copy of this 
% license, visit http://creativecommons.org/licenses/by-nc-sa/2.5/ or send a 
% letter to Creative Commons, 543 Howard Street, 5th Floor, San Francisco, 
% California, 94105, USA.

\documentclass[letterpaper,11pt]{article}
\newlength{\outerbordwidth}
\pagestyle{empty}
\raggedbottom
\raggedright
\usepackage[svgnames]{xcolor}
\usepackage{framed}
\usepackage{tocloft}
\usepackage[hidelinks]{hyperref}

%-----------------------------------------------------------
%Edit these values as you see fit

\setlength{\outerbordwidth}{3pt}  % Width of border outside of title bars
\definecolor{shadecolor}{gray}{0.75}  % Outer background color of title bars 
                                      % (0 = black, 1 = white)
\definecolor{shadecolorB}{gray}{0.93}  % Inner background color of title bars


%-----------------------------------------------------------
%Margin setup

\setlength{\evensidemargin}{-0.25in}
\setlength{\headheight}{0in}
\setlength{\headsep}{0in}
\setlength{\oddsidemargin}{-0.25in}
\setlength{\paperheight}{11in}
\setlength{\paperwidth}{8.5in}
\setlength{\tabcolsep}{0in}
\setlength{\textheight}{9.5in}
\setlength{\textwidth}{7in}
\setlength{\topmargin}{-0.3in}
\setlength{\topskip}{0in}
\setlength{\voffset}{0.1in}


%-----------------------------------------------------------
%Custom commands
\newcommand{\resitem}[1]{\item #1 \vspace{-2pt}}
\newcommand{\resheading}[1]{\vspace{8pt}
  \parbox{\textwidth}{\setlength{\FrameSep}{\outerbordwidth}
    \begin{shaded}
\setlength{\fboxsep}{0pt}\framebox[\textwidth][l]{\setlength{\fboxsep}{4pt}\fcolorbox{shadecolorB}{shadecolorB}{\textbf{\sffamily{\mbox{~}\makebox[6.762in][l]{\large #1} \vphantom{p\^{E}}}}}}
    \end{shaded}
  }\vspace{-5pt}
}
\newcommand{\ressubheading}[4]{
\vspace{10pt}
\begin{tabular*}{6.5in}{l@{\cftdotfill{\cftsecdotsep}\extracolsep{\fill}}r}
		\textbf{#1} & #2 \\
		\textit{#3} & \textit{#4} \\
\end{tabular*}\vspace{-6pt}
\vspace{10pt}}
%-----------------------------------------------------------


\begin{document}

\begin{tabular*}{7in}{l@{\extracolsep{\fill}}r}
\textbf{\Large Alexander D Brown} & alex@alexanderdbrown.com \\
Little Barmoor, \\Sunset Lane, \\West Chiltington, \\West Sussex, \\RH20 2NY & \url{http://alexanderdbrown.com} \\
\end{tabular*}
\\

\vspace{10pt}
\textbf{I am an IEEE published author studying Software Engineering at
Masters-level with an enthusiasm for teaching.}

\textbf{In previous workplaces I have often been praised for my enthusiasm and 
can-do attitude; often demonstrating I was able to work well independently, 
with a wide range of people; from school children to senior members of 
management.}


%%%%%%%%%%%%%%%%%%%%%%%%%%%%%%
\resheading{Education}
%%%%%%%%%%%%%%%%%%%%%%%%%%%%%%
\ressubheading{Aberystwyth University}{Aberystwyth}{Master of Engineering 
(MEng)  Software Engineering}{2009 - 2014}

% Awards
Awarded the Portaltech Reply Bursary in Computer Science for best performance
in the penultimate year of the MEng scheme.

% Marks
% Currently achieving a mark of a first/high 2:1 after four years of study, 
% including a single year of industry at IBM Hursley.

% Disertation and Paper
Produced a highly marked (78\%) third year project entitled 
``Kyffin Williams: Digital Analysis of Paintings'', which also resulted in a 
paper for the \textbf{8\textsuperscript{th} International Symposium on Image 
and Signal Process and Analysis (ISPA)} entitled 
``Can we date an artists work from catalogue photographs?'' 

% Presenting paper
This paper was later presented at the conference in Trieste on 
3\textsuperscript{rd} September, 2013 and was co-authored with the project 
supervisor, Hannah Dee; a PhD student at the Aberystwyth School of Art, Gareth 
Lloyd Roderick; and a Professor of Digital Humanities at the National Library 
of Wales, Lorna M. Hughes.

% BSc Show and Tell
This third year project was also presented the British Computer 
Society (BCs) Show and Tell event twice at Aberystwyth University, gaining good
feedback from members of the audience including members of staff in the 
Computer Science department.

% Gregynog
Produced and presented on the topic of industrial years and improving social
media presence at a careers weekend run by the department to small groups of
second year students intending to take industrial years themselves in November
2013, 2012 and 2011.

% Technocamps
Volunteered as part of the Technocamps project at the university, teaching 
school children aged 11-15 year basic electronics and programming for them to 
be able to build semi-automated robots.

% Projects
Have produced a number of programs relating to a range of subjects including: 
implementing Artificial Intelligence algorithms, developing RESTful services,
solving complex problems and mobile development for both websites and native 
applications


%%%%%%%%%%%%%%%%%%%%%%%%%%%%%%
\resheading{Work Experience}
%%%%%%%%%%%%%%%%%%%%%%%%%%%%%%

\ressubheading{Aberystwyth University}{Department of Computer Science}{Advisor}
{September 2013 - Present}

Responsible for running a drop-in service to help students with understanding
course material or extra-curricular projects. The majority of problems were
related to debugging a variety of languages including PHP, C and Java. Other
problems included software installation and theoretical understanding of 
programming paradigms.

Organised and taught half of a two day course to introduce first year students
to basic programming concepts including simple data structures common to most
programming languages and flows of control.

Responsible for the sign-off of assessed worksheets in the practical sessions
for three first year modules. The first of these modules
focused on the use of the UNIX command-line environment, the second of these
modules taught development of basic Java and Haskell applications and the third
Java abstraction, GUIs in Swing and threading.

\ressubheading{IBM}{IBM Hursley}{CICS Level 3 Service Tooling Engineer}
{June 2012 - July 2013}

% Tooling responsibilities
Responsible for designing and developing useful Java-based tools for the CICS 
Level 3 Service team, including an eclipse plug-in to print out information 
required for code reviews and a large system to automate the delivery of fix 
patches for CICS Eclipse-based products, which hooked into many internal 
systems.

% RTC responsibilities
Helped gather requirements to apply to a system designed to be used by all 
Level 3 Service teams so that the CICS team would not be disrupted in their 
work and attended meetings to discuss the development of this system.

% Maintenance
Maintained and improved several systems for generating statistics for problem 
reports and the processes for fixing these problems, including a Java 
Enterprise server and DB2 database hosted on a CentOS Enterprise Linux server.

% Java Masterclass
Lead a team of three IBM employees to run a Java-master at Swanmore School of 
Technology, to get school children aged 13-15 years introduced to programming 
in the Java programming language at a basic level.

% Teaching
Helped teach several Java sessions internally within IBM to help members of the
Level 3 CICS Service team and Industrial Trainees gain the skills and knowledge
needed to use Java in their jobs. Mentored by an ex-lecturer from the 
University of Southampton to help decide the content of these sessions and the 
teaching style involved.

% IT Inductions
Organised the inductions for the 2012-13 intake of Industrial Trainees for 
their first two days at IBM, requiring the networking with both managers of 
each trainee to ensure they had the equipment and logins for their roles, as 
well as members of upper-management to present introductory talks at each of 
the three inductions. This was done as part of a team of three, with the help 
of industrial trainee and graduate managers.

% Blue Fusion
Designed, built and tested a game based on the travelling salesman problem for
school children to play on the Blue Fusion event, run over national science 
week. Worked as part of a team of five using Java with the AWT graphical 
library.

This game was also used as the event's finale, in which the AI agents the 
players produced were pitted against one another to determine which team had the
best algorithm.


\ressubheading{Aberystwyth University}{Department of Computer Science}{Part 
Time Teacher/Demonstrator}{September 2011 - May 2012}

% Responsibilities
Responsible for the sign-off of assessed worksheets in the practical sessions 
for two first year modules. The first of these modules 
focused on the use of the UNIX command-line environment and the second
modules taught development of Java applications, particularly GUI-based 
applications using the swing libraries and multi-threaded programming 
paradigms.

% Excellence
In the December 2011 Staff Student Committee Meeting I was praised as an 
``excellent'' demonstrator.

% Peer observation of teaching
Contributed to the peer observation of teaching for demonstrators, giving 
feedback on how to improve the teaching methods of demonstrators and 
the process of the demonstration of practical sessions.


%%%%%%%%%%%%%%%%%%%%%%%%%%%%%%
\resheading{Programming Experience}
%%%%%%%%%%%%%%%%%%%%%%%%%%%%%%

\ressubheading{Examples}{}{}{}

Code for the following projects and other, smaller projects, has been made 
available at: \url{https://github.com/SoftlySplinter/} where possible.

\ressubheading{Java}{}{}{}

% OSGi and Eclipse plug-ins
Used OSGi to develop a highly-modular Java system which automated the process
of releasing fix packs of Eclipse-based tools, integrating several external 
services and APIs, some of which were known to change their API without 
warning. The purpose of using OSGi for this task was to allow the whole system
to continue functioning if one service changed in a way which broke the code,
or was simply unavailable at that time.

Have also developed Eclipse plug-ins, again using the OSGi framework, to improve
workflow of the CICS Level 3 Service team.

% Flocking Simulator
Developed a simple program to simulate the flocking behaviour of birds and to
show this behaviour graphically using the AWT graphical libraries. 

% Java EE
Maintained and improved a Java EE application to automate the gathering of 
statistics running on a IBM Webspehere Server and using Java Beans to connect to
a IBM DB2 database.

% Group project - RoSe
Building a Java EE broker, as part of an assignment, which stores hotel and 
room booking information and serves them through a SOAP web service.

\ressubheading{Python}{}{}{}

%TODO
% pytentd
Developed an open source python implementation of the tent
protocol using the Flask microframework, with automated testing through Travis.
The main focus was on the REST API for the server to meet the specification.

% Dissertation
Built a command-line based tool which used OpenCV to perform a number of 
image processing analysis techniques, including colour-space analysis; texture
analysis and histogram comparison, on a set of pictures. This analysis was 
then fed into a classification algorithm and used to perform validation of how
well each analysis technique performed using statistical correlation.

% Genetic Algorithm
Produced a command-line tool which used a genetic algorithm to solve the 
travelling salesman problem using a number of different crossover and mutation
operators


\ressubheading{C}{}{}{}

% Compiler for armok
Built an interpreter for an esoteric programming language based on the dwarf
fortress game using the standard GCC libraries and including support for UTF8
character encoding. This also included a small suite of tests to ensure the
language functioned correctly and Travis to automate these tests.


\ressubheading{PHP}{}{}{}

Built basic PHP server-side scripts to display certain dynamic content on my
personal website, using object-orientated PHP, including namespaces.


\ressubheading{Ruby}{}{}{}

Built a Rails server and desktop application which communicated using REST which
included testing using RSpec.


\ressubheading{Objective C}{}{}{}

% Mobile development for iOS
Built a simple iOS application designed to store and teach Welsh vocabulary,
including memorisation games.

\ressubheading{C\#}{}{}{}

% Service client
Building a .NET server which connects to a Java broker via SOAP web services to
send and receive data.


\ressubheading{Scala}{}{}{}

Participated in the Functional Programming Principles in Scala course online
which taught basic Scala, including set operation, filtering, sorting
information based on keywords and Huffman coding.


%%%%%%%%%%%%%%%%%%%%%%%%%%%%%%
\resheading{Volunteering}
%%%%%%%%%%%%%%%%%%%%%%%%%%%%%%

\ressubheading{Black House}{Aberystwyth Students Union}{Performance
Artist}{March 2014}

Performed circus skills for the Black House 2014 event, including staff
manipulation, juggling, fire staff, fire poi and contact poi.


\ressubheading{Aberystwyth Univeristy}{}{Demonstrator}{February 2014}

Assisted with a STEM event held by Aberystwyth University which taught year 9
school children basic Linux terminal skills.

More information:
\url{http://blog.alexanderdbrown.com/stem-hack-the-gibson.html}


\ressubheading{West Wales Linux Users Group}{}{}{September 2013 - present}

Organised and talked at two events for the West Wales LUG, one focusing on
installing Linux on new students laptops and one on Linux editors.


\ressubheading{Technocamps}{Technocamps Aberystwyth}{Support Staff}{October
2012 - December 2012}

% Responsibilities
Attended weekly club sessions to supervise activities for school children 
ages 11 to 19, which involved teaching the Arduino electronics platform, 
robotics and basic software development.


\ressubheading{Battle for Wesnoth}{}{Pixel Artist}{February 2009 - July 2010}

Worked as part of a team of pixel artists to improve the art for the open 
source game ``Battle for Wesnoth'', which involved producing animation frames 
as well as giving and receiving critique.


\ressubheading{Adur Canoe Club}{}{Webmaster}{September 2009 - August 2010}

% Responsibilities
Redesigned and maintained the canoe clubs website to improve the user 
experience. Primarily working with HTML, CSS and JavaScript, but also producing
small PHP scripts based on the existing content to improve some of the 
functionality.

The current version of the website can be found at 
\url{http://adurcanoeclub.org.uk/} and although the design has changed slightly
since the new webmaster has taken over, it still retains many similar elements.

% Committee
Also served as a committee member during this time, attending regular meetings
to organise the running of the club.

\ressubheading{Adur Canoe Club}{}{Instructor}{September 2009 - August 2010}

%TODO
% Responsibilities
Acted as the duty instructor for several club sessions, taking responsibility 
for the planning of the session and the safety of the kayakers on the water.
Also acted as support for other duty instructors, either taking groups or
acting as safety cover.

% Coaches Meeting
Also attended a meeting between all coaches in the club to improve the teaching
done by all coaches and giving general feedback to individual coaches.


%%%%%%%%%%%%%%%%%%%%%%%%%%%%%%
\resheading{Server Administration}
%%%%%%%%%%%%%%%%%%%%%%%%%%%%%%

\ressubheading{Debian Server}{http://alexanderdbrown.com/}{Debian 6.0.7}{2011 - present}

% Personal VPS
Administered a personal VPS running a variety of services, including Nginx to
serving static content as well as PHP using PHP-FPM (FastCGI Process Manager)
and a WSGI implementation for testing Flask applications on a real server.

% Python-based blog
Primarily used for hosting a blog using the Python-based Pelican framework, 
which I have begun to contribute to.


\ressubheading{CentOS Server}{IBM CICS L3 Service Statistics Server}{}{2012 - 2013}

% Java EE WebSphere Server
% DB2 Database
Maintained and administered an in-house server which ran a Java EE WebSphere 
server which managed the automated collection of statistics for the CICS L3 
Service management. This server also hosted a DB2 database for persisting this
information.


\ressubheading{Ubuntu-based Server}{IBM CICS L3 Service Sandbox}{}{2013}

% Testing Server
Set up and maintained a testing server to help determine how the future 
versions of certain APIs would affect the CICS L3 Service tooling and to be
able to react to such changes before the changes were implemented on the 
in-house servers.

% RTC Build Engines
This server was also used to host backup build engines for the in-house servers
to allow some redundancy in the system.

% RTC Sandbox Server 
The other use of this server was to test out the capabilities of what the 
system could do and how the work flow of the team could be improved through the
use of newer features of this system.


%%%%%%%%%%%%%%%%%%%%%%%%%%%%%%%%%%
\resheading{Interests and Hobbies}
%%%%%%%%%%%%%%%%%%%%%%%%%%%%%%%%%%

I am an amateur photographer, my work can be found here: 
\url{http://www.flickr.com/photos/softlysplinter/}.

\hfill \\

I am a BCU qualified Level 2 Kayaking Instructor, with an interest in 
whitewater kayaking.

\hfill \\

I play the drums and have achieved Grade 4.

\hfill \\
I enjoy learning circus skills, including juggling, poi and staff manipulation
and have performed for the Black House 2014 event.

\hfill \\

I am beginning to learn the Iaido martial art through the Bujin-Ryu school of 
Iaido and Iaijustsu.

\resheading{References}

References available on request.


\end{document}
