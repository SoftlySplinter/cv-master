% (c) 2002 Matthew Boedicker <mboedick@mboedick.org> (original author) 
% http://mboedick.org
% (c) 2003-2007 David J. Grant <davidgrant-at-gmail.com> 
% http://www.davidgrant.ca
% (c) 2008 Nathaniel Johnston <nathaniel@nathanieljohnston.com> 
% http://www.nathanieljohnston.com
%
% (c) 2012 Scott Clark <sc932@cornell.edu> cam.cornell.edu/~sc932
%
%This work is licensed under the Creative Commons 
% Attribution-Noncommercial-Share Alike 2.5 License. To view a copy of this 
% license, visit http://creativecommons.org/licenses/by-nc-sa/2.5/ or send a 
% letter to Creative Commons, 543 Howard Street, 5th Floor, San Francisco, 
% California, 94105, USA.

\documentclass[letterpaper,11pt]{article}
\newlength{\outerbordwidth}
\pagestyle{empty}
\raggedbottom
\raggedright
\usepackage[svgnames]{xcolor}
\usepackage{framed}
\usepackage{tocloft}


%-----------------------------------------------------------
%Edit these values as you see fit

\setlength{\outerbordwidth}{3pt}  % Width of border outside of title bars
\definecolor{shadecolor}{gray}{0.75}  % Outer background color of title bars 
                                      % (0 = black, 1 = white)
\definecolor{shadecolorB}{gray}{0.93}  % Inner background color of title bars


%-----------------------------------------------------------
%Margin setup

\setlength{\evensidemargin}{-0.25in}
\setlength{\headheight}{0in}
\setlength{\headsep}{0in}
\setlength{\oddsidemargin}{-0.25in}
\setlength{\paperheight}{11in}
\setlength{\paperwidth}{8.5in}
\setlength{\tabcolsep}{0in}
\setlength{\textheight}{9.5in}
\setlength{\textwidth}{7in}
\setlength{\topmargin}{-0.3in}
\setlength{\topskip}{0in}
\setlength{\voffset}{0.1in}


%-----------------------------------------------------------
%Custom commands
\newcommand{\resitem}[1]{\item #1 \vspace{-2pt}}
\newcommand{\resheading}[1]{\vspace{8pt}
  \parbox{\textwidth}{\setlength{\FrameSep}{\outerbordwidth}
    \begin{shaded}
\setlength{\fboxsep}{0pt}\framebox[\textwidth][l]{\setlength{\fboxsep}{4pt}\fcolorbox{shadecolorB}{shadecolorB}{\textbf{\sffamily{\mbox{~}\makebox[6.762in][l]{\large #1} \vphantom{p\^{E}}}}}}
    \end{shaded}
  }\vspace{-5pt}
}
\newcommand{\ressubheading}[4]{
\begin{tabular*}{6.5in}{l@{\cftdotfill{\cftsecdotsep}\extracolsep{\fill}}r}
		\textbf{#1} & #2 \\
		\textit{#3} & \textit{#4} \\
\end{tabular*}\vspace{-6pt}}
%-----------------------------------------------------------


\begin{document}

\begin{tabular*}{7in}{l@{\extracolsep{\fill}}r}
\textbf{\Large Alexander D Brown} & alex@alexanderdbrown.com \\
Little Barmoor, \\Sunset Lane, \\West Chiltington, \\West Sussex, \\RH20 2NY & http://alexanderdbrown.com \\
\end{tabular*}
\\

\vspace{10pt}
\textbf{I am an IEEE published author studying Software Engineering at
Masters-level with an enthusiasm for teaching.}

\textbf{In previous workplaces I have often been praised for my enthusiasm and 
can-do attitude; often demonstrating I was able to work well independently, as 
well as with a wide range of people; from school children to senior members of 
management.}


%%%%%%%%%%%%%%%%%%%%%%%%%%%%%%
\resheading{Education}
%%%%%%%%%%%%%%%%%%%%%%%%%%%%%%
\ressubheading{Aberystwyth University}{Aberystwyth}{Master of Engineering 
(MEng)  Software Engineering}{2009 - 2014}
\vspace{10pt}

% Marks
Currently achieving a mark of a first/high 2:1 after four years of study, 
including a single year of industry at IBM Hursley.

% Disertation and Paper
Produced a highly marked (78\%) third year project entitled 
``Kyffin Williams: Digital Analysis of Paintings'', which also resulted in a 
paper for the \textbf{8\textsuperscript{th} International Symposium on Image 
and Signal Process and Analysis (ISPA)} entitled 
``Can we date an artists work from catalogue photographs?'' 

% Presenting paper
This paper was later presented at the conference in Trieste on 
3\textsuperscript{rd} September, 2013 and was co-authored with the project 
supervisor, Hannah Dee; a PhD student at the Aberystwyth School of Art, Gareth 
Lloyd Roderick; and a Professor of Digital Humanities at the National Library 
of Wales, Lorna M. Hughes.

% BSc Show and Tell
This third year project was also presented twice at the British Computer 
Society (BCs) Show and Tell event twice at Aberystwyth University, gaining good
feedback from members of the audience including members of staff in the 
Computer Science department.

% Gregynog
Produced and presented on the topic of industrial years and improving social
media presence at a careers weekend run by the department to small groups of
second year students intending to take industrial years themselves in November
2013, 2012 and 2011.

% Technocamps
Volunteered as part of the Technocamps project at the university, teaching 
school children aged 11-15 year basic electronics and programming for them to 
be able to build semi-automated robots.

% Projects
Have produced a number of programs relating to a range of subjects including; a
genetic algorithm to solve the travelling salesman problem in Python, 
developing clients for RESTful services in Ruby and solving Sudoku puzzles 
in Java.

Have researched into several subjects including genetic algorithms, image
processing filters; specifically Gabor filters, the domain name system and
agile methodologies.

%%%%%%%%%%%%%%%%%%%%%%%%%%%%%%
\resheading{Work Experience}
%%%%%%%%%%%%%%%%%%%%%%%%%%%%%%

\vspace{10pt}
\ressubheading{Aberystwyth University}{Department of Computer Science}{Advisor}
{2013-2014}
\vspace{10pt}

Responsible for running a drop-in service to help students with understanding
course material or extra-curricular projects. Problems range from debugging to
software installation to theoretical understanding of computer science.

Responsible for the sign-off of assessed worksheets in the practical sessions
for the ``Introduction to Computer Operating Systems, Hardware and UNIX Tools''
and ``Concepts in Programming'' first year modules. The first of the modules
focused on the use of the UNIX command-line environment and the second of these
modules taught development of basic Java and Haskell applications.

\vspace{10pt}
\ressubheading{IBM}{IBM Hursley}{CICS Level 3 Service Tooling Engineer}
{2012-2013}
\vspace{10pt}

% Tooling responsibilities
Responsible for designing and developing useful Java-based tools for the CICS 
Level 3 Service team, including an eclipse plug-in to print out information 
required for code reviews and a large system to automate the delivery of fix 
patches for CICS Eclipse-based products, which hooked into many internal 
systems.

% RTC responsibilities.
Helped gather requirements to apply to a system designed to be used by all 
Level 3 Service teams so that the CICS team would not be disrupted in their 
work and attended meeting to discuss the development of this system.

% Maintenance
Maintained and improved several systems for generating statistics for problem 
reports and the processes for fixing these problems, including a Java 
Enterprise server and DB2 database hosted on a CentOS Enterprise Linux server.

Lead a team of three IBM employees to run a Java-master at Swanmore School of 
Technology, to get school children aged 13-15 years introduced to programming 
in the Java programming language at a basic level.

% Teaching
Helped teach several Java sessions internally within IBM to help members of the
Level 3 CICS Service team and Industrial Trainees gain the skills and knowledge
needed to use Java in their jobs. Mentored by an ex-lecturer from the 
University of Southampton to help decide the content of these sessions and the 
teaching style involved.

% IT Inductions
Organised the inductions for the 2012-13 intake of Industrial Trainees for 
their first two days at IBM, requiring the networking with both managers of 
each trainee to ensure they had the equipment and logins for their roles, as 
well as members of upper-management to present introductory talks at each of 
the three inductions. This was done as part of a team of three, with the help 
of industrial trainee and graduate managers.

% Blue Fusion
%TODO


\vspace{10pt}
\ressubheading{Aberystwyth University}{Department of Computer Science}{Part 
Time Teacher/Demonstrator}{2011-2012}
\vspace{10pt}

% Responsibilities
Responsible for the sign-off of assessed worksheets in the practical sessions 
for the ``Introduction to Computer Operating Systems, Hardware and UNIX Tools''
and ``Software Development'' first year modules. The first of the modules 
focused on the use of the UNIX command-line environment and the second of these
modules taught development of Java applications, particularly GUI-based 
applications using the swing libraries and multi-threaded programming 
paradigms.

% Excellence
In the December 2011 Staff Student Committee Meeting I was praised as an 
``excellent'' demonstrator.

% Peer observation of teaching
Contributed to the peer observation of teaching for demonstrators, giving 
feedback on how to improve the teaching methods of demonstrators and to improve
the process of the demonstration of practical sessions.


\resheading{Skills and Experience}

\vspace{-16pt}
\begin{description}
\item[Teaching] \hfill \\
Qualified British Canoe Union (BCU) Level 2 Kayaking Instructor. Ran a number 
of sessions as the duty instructor for Adur Canoe Club, as well as supporting 
other instructors as part of their duty sessions, involving teaching kayakers 
of a wide range of ages and skill levels the skills required to improve their 
paddling.

\item[Communication and Presentation] \hfill \\
Attended presentation workshops in the business centre of IBM Hursley to 
improve presentation skills with the aid of professional presenters.

Presented a range of topics to several audiences including the findings of a 
paper at a conference, the teaching of computer science and kayaking techniques
and the application image processing techniques at the BSc Show and Tell.

Announced and moderated talks at the Free and Open Source Developers Meet 
(FOSDEM) in Brussels, Belgium in 2011.

\item[Computing] \hfill \\
Good knowledge of the Java programming language (versions 1.6 and 1.7) having 
attended sessions to become certified in Java 7 and produced many Java 
projects, including a large scale system which hooked into several other 
systems using varying APIs, a simulation of flocking behaviour and a game to 
teach children mathematical and technological concepts by programming an agent 
to visit locations in the shortest distance.

Good knowledge of the UNIX/Linux command-line, having demonstrated the topic 
and having managed several servers both personal and within IBM.

%TODO this needs more detail by far.

\end{description}

% Communication
%   Paper Presentation
%   Presentation Workshops
%   FOSDEM talk moderations
% Organisation
%   Swanmore, IT Inductions
%   Projects at IBM
% Skills
%   Java
%     OSGi, Eclipse, GUI, etc.
%   Python
%     OpenCV et al.
%   Ruby, perl, PHP, JS, 

\end{document}
